\chapter{Playing Games}
One major success for the field of artificial intelligence happened in 1997 when the chess-playing computer
\href{https://en.wikipedia.org/wiki/Deep_Blue_(chess_computer)}{Deep Blue} was able to beat the World Chess
Champion \href{https://en.wikipedia.org/wiki/Garry_Kasparov}{Garry Kasparov} by $3\sfrac{1}{2}-2\sfrac{1}{2}$.
While \emph{\color{blue}Deep Blue} was based on special hardware, today according to the
\href{http://www.computerchess.org.uk/ccrl/4040/rating_list_all.html}{computer chess rating list} 
top performing chess programs like \href{https://en.wikipedia.org/wiki/Stockfish_(chess)}{Stockfish} have an 
\href{https://en.wikipedia.org/wiki/Elo_rating_system}{Elo rating} of 3392.  To compare, according to 
\href{https://ratings.fide.com/top.phtml?list=men}{Fide}, the current 
World Chess Champion \href{https://en.wikipedia.org/wiki/Magnus_Carlsen}{Magnus Carlsen} has an Elo rating of
just 2838.  Hence, he wouldn't stand a chance to win a game against Stockfish.  More recently, the computer program
\href{https://en.wikipedia.org/wiki/AlphaGo}{AlphaGo} was able to beat
\href{https://en.wikipedia.org/wiki/Lee_Sedol}{Lee Sedol}, who is considered to be the second best 
\href{https://en.wikipedia.org/wiki/Go_(game)}{go} player in the world.  Besides go and chess, there are many
other games where today the performance of a computer exceeds the performance of human players.  To name just
one more example, at the beginning of 2017 the program \href{https://en.wikipedia.org/wiki/Libratus}{Libratus} was able to 
\href{https://www.engadget.com/2017/01/31/libratus-the-poker-playing-ai-destroyed-its-four-human-rivals/}{beat}
four professional poker players resoundingly.

In this chapter we want to investigate how a computer can play a game.  To this end we define a
\blue{game} $\mathcal{G}$ as a six-tuple
\\[0.2cm]
\hspace*{1.3cm}
$\mathcal{G} = \langle \mathtt{States}, s_0, \mathtt{Players}, \mathtt{nextStates}, \mathtt{finished},\mathtt{utility} \rangle$
\\[0.2cm]
where the components are interpreted as follows:
\begin{enumerate}
\item $\mathtt{States}$ is the set of all possible \emph{\color{blue}states} of the game.
\item $s_0 \in \mathtt{startState}$ is the \emph{\color{blue}start state}.
\item $\mathtt{Players}$ is  the set of \blue{players} of the game.
\item $\mathtt{nextStates}$ is a function that takes a state and a player $p$ and returns the set of
      states that can be reached if $p$ has to make a move in the game.  Hence, the signature of
      $\mathtt{nextStates}$ is given as follows:
      \\[0.2cm]
      \hspace*{1.3cm}
      $\mathtt{nextStates}: \mathtt{State} \times \mathtt{Player} \rightarrow 2^{\mathtt{States}}$.
\item $\mathtt{finished}$ is a function that takes a state $s$ and decides whether the games is finished.
      Therefore
      \\[0.2cm]
      \hspace*{1.3cm}
      $\mathtt{finished}: \mathtt{States} \rightarrow \mathbb{B}$.
      \\[0.2cm]
      Here, $\mathbb{B}$ is the set of Boolean values, i.e.~we have $\mathbb{B} := \{ \mathtt{true}, \mathtt{false} \}$.
  
      Using the function $\mathtt{finished}$, we define the set $\mathtt{TerminalStates}$ as the set of those
      states such that the game has finished,  i.e.~we define
      \\[0.2cm]
      \hspace*{1.3cm}
      $\mathtt{TerminalStates} := \{ s \in \mathtt{States} \mid \mathtt{finished}(s) \}$.
\item $\mathtt{utility}$ is a function that takes a state $s \in \mathtt{TerminalStates}$ and a player $p$.  If returns
      the \blue{value} that the game has for player $p$.  In all of our examples, this value will be an element
      from the set $\{-1, 0, +1\}$.  The value $-1$ indicates that player $s$ has lost the game,
      if the value is $+1$ the player $p$ has won the game and if this value is $0$ then the game is drawn.
      Hence the signature of $\mathtt{utility}$ is
      \\[0.2cm]
      \hspace*{1.3cm}
      $\mathtt{utility}: \mathtt{TerminalStates} \times \mathtt{Players} \rightarrow \{ -1, 0, +1\}$.
\end{enumerate}

\exercise
The definition given above does not capture all types of games.  In particular, the definition only captures 
\blue{deterministic games}:  A game is called \blue{deterministic} iff there is no randomness
involved.  Games like chess and go are certainly deterministic.  However, other games like 
\href{https://en.wikipedia.org/wiki/Dice_chess}{dice chess} involve randomness.  In dice chess a pair of dice is
rolled at the beginning of each turn.  The outcome of this roll determines which pieces may be moved.
Extend the definition of a
game so that also a games like \blue{dice chess} be modelled.  
\eox

In this chapter we will only consider so called \blue{two person zero sum games}.  This means that the set $\mathtt{Players}$
has exactly two elements.  If we call these players $\mathrm{A}$ and $\mathrm{B}$, i.e.~if we have
\\[0.2cm]
\hspace*{1.3cm}
$\mathtt{Players} = \{ \mathrm{A}, \mathrm{B} \}$,
\\[0.2cm]
then the game is called a \blue{zero sum game} iff we have
\\[0.2cm]
\hspace*{1.3cm}
$\forall s \in \mathtt{TerminalStates}:\mathtt{utility}(s, \mathtt{A}) + \mathtt{utility}(s, \mathtt{B}) = 0$,
\\[0.2cm]
i.e.~the losses of player $\mathrm{A}$ are compensated by the wins of player $\mathtt{B}$ and vice versa.
Games like \href{https://en.wikipedia.org/wiki/Go_(game)}{go} and 
\href{https://en.wikipedia.org/wiki/Chess}{chess} are two person zero sum games.

\example
The game \href{https://en.wikipedia.org/wiki/Tic-tac-toe}{tic-tac-toe} is played on a square board of size 
$3 \times 3$.  On every turn, player $\mathrm{A}$ puts an ``\texttt{X}'' on one of the free squares of the board, while
player $\mathrm{B}$ puts an $\mathtt{O}$ onto a free square.  If player $\mathrm{A}$ has three \texttt{X}s in a
row, column, or diagonal, he has won the game.  Similarly, if player $\mathrm{B}$ has three \texttt{O}s in a
row, column, or diagonal, player $\mathrm{B}$ is the winner.  Otherwise, the game is drawn.

\begin{figure}[!ht]
\centering
\begin{Verbatim}[ frame         = lines, 
                  framesep      = 0.3cm, 
                  firstnumber   = 1,
                  labelposition = bottomline,
                  numbers       = left,
                  numbersep     = -0.2cm,
                  xleftmargin   = 0.8cm,
                  xrightmargin  = 0.8cm,
                ]
    players := procedure() { return { "X", "O" }; };
    startState := procedure(n) {
        L := [1 .. n];
        return [ [ " " : col in L] : row in L];
    };
    nextStates := procedure(State, player) {
        n      := #State;
        L      := [1 .. n];
        Empty  := find_empty(State);
        Result := {};
        for ([row, col] in Empty) {
            NextState           := State;
            NextState[row][col] := player;
            Result              += { NextState };
        }
        return Result;
    };
    find_empty := procedure(State) {
        n := #State;
        L := [1 .. n];
        return { [row, col] : row in L, col in L | State[row][col] == " " };
    };
    all_lines := procedure(State) {
        n := #State;
        L := [1 .. n];
        Lines := { { State[row][col] : col in L } : row in L };
        Lines += { { State[row][col] : row in L } : col in L };
        Lines += { { State[idx][idx] : idx in L } };
        Lines += { { State[idx][n - (idx - 1) ] : idx in L } };
        return Lines;
    };
    utility := procedure(State, player) {
        Lines := all_lines(State);
        for (line in Lines) {
            if (#{ x : x in line } == 1 && line != { " " }) {
                if (line == { player }) { return  1; } else { return -1; }
            }
        }
        if (find_empty(State) == {}) { return 0; }
    };
    finished := procedure(State) {
        player := "X";
        if (utility(State, player) != om) { return true; } else { return false; }
    };
\end{Verbatim}
\vspace*{-0.3cm}
\caption{A \textsc{SetlX} description of tic-tac-toe.}
\label{fig:ttt.stlx}
\end{figure}
 
\myFig{ttt.stlx} shows a \textsc{SetlX} implementation of tic-tac-toe.
\begin{enumerate}
\item The function $\mathtt{players}$ returns the set $\mathtt{Players}$.  Traditionally, the players in
      tic-tac-toe are called ``\texttt{X}'' and ``\texttt{O}''.
\item The function $\mathtt{startState}$ returns the start state, which is an empty board.
      States are represented as list of lists.  The entries in these lists are the characters 
      ``\texttt{X}'', ``\texttt{O}'', and ``\texttt{ }''.
      As the  $\mathtt{StartState}$ is the empty board, it is represented as as follows:
      \begin{Verbatim}
      [ [" ", " ", " "], 
        [" ", " ", " "], 
        [" ", " ", " "]
      ].     
      \end{Verbatim}
      The function $\mathtt{startState}$ receives one optional argument $\mathtt{n}$.
      By default $\mathtt{n}$ is equal to $3$.  This argument specifies the size of the board.
      Although traditionally tic-tac-toe is played on a $3 \times 3$ board, it can also 
      be played on a bigger board.
\item The function $\mathtt{nextStates}$ takes a $\mathtt{State}$ and a $\mathtt{player}$ and computes the set
      of states that can be reached from $\mathtt{State}$ if $\mathtt{player}$ is to move next.
\end{enumerate}





%%% Local Variables:
%%% mode: latex
%%% TeX-master: "artificial-intelligence"
%%% End:
