\chapter{Introduction}

\section{What is Artificial Intelligence?}
Before we start to dive into the subject of
\href{https://en.wikipedia.org/wiki/Artificial_intelligence}{Artificial Intelligence} we have to answer the
following question:  
\\[0.2cm]
\hspace*{1.3cm}
What is \emph{Artificial Intelligence}? 
\\[0.2cm] 
Historically, there have been a number of different answers to this question \cite{russell:2009}.  We will look at these
different answers and discuss them.
\begin{enumerate}
\item \emph{Artificial Intelligence} is the study of creating machines that \underline{think} like humans.

      As we have a working prototype of intelligence, namely humans, it is quite natural to try to
      build machines that work in a way similar to humans, thereby creating artificial
      intelligence.  As a first step in this endeavor we would have to study how humans actually
      think and thus we would have to study the brain.  Unfortunately, as of today, no one really
      knows how the brain works.  Although there are branches of science devoted to studying the
      human thought processes and the human brain, namely
      \href{https://en.wikipedia.org/wiki/Cognitive_science}{cognitive science} and 
      \href{https://en.wikipedia.org/wiki/Computational_neuroscience}{computational neuroscience},
      this approach has not proven to be fruitful for creating thinking machines, the reason being
      that the current understanding of the human thought processes is just not sufficient.  

\item \emph{Artificial Intelligence} is the science of machines that \underline{act} like people.

      Since we do not know how humans think,  we cannot build machines that think like people.
      Therefore, the next best thing might be to build  machines that act and behave like humans.  
      Actually, the \href{https://en.wikipedia.org/wiki/Turing_test}{Turing Test} is based on this
      idea:  Turing suggested that if we want to know whether we have succeeded in building an
      intelligent machine, we should place it at the other end of a chat line.  If we cannot
      distinguish the computer from a human, then we have succeeded at creating intelligence. 

      However, with respect to the kind of Artificial Intelligence that is needed in industry, this
      approach isn't very useful.  To illustrate the point, consider an analogy with aerodynamics:
      In aerodynamics we try to build planes that fly fast and efficiently, not planes that flap
      their wings like birds do, as the later approach has failed historically, e.g. 
      \href{https://en.wikipedia.org/wiki/Daedalus#Daedalus_and_Icarus}{Daedalus and Icarus}.
\item \emph{Artificial Intelligence} is the science of creating machines that \underline{think lo}g\underline{icall}y.

      The idea with this approach is to create machines that are based on 
      \href{https://en.wikipedia.org/wiki/Mathematical_logic}{mathematical logic}.  If a
      goal is given to these machines, then these machines use logical reasoning in order to deduce
      those actions that need to be performed in order to best achieve the given goals. 
      Unfortunately, this approach had only limited success:  In playing
      games the approach was quite successful for dealing with games like checkers or chess.
      However, the approach was mostly unsuccessful for dealing with many real world problems.
      There were two main reasons for its failure:
      \begin{enumerate}
      \item In order for the logical approach to be successful, the environment has to be
            \emph{\color{blue}completely} 
            described by mathematical axioms.  It has turned out that our knowledge of the real
            world is often not sufficient to completely describe the environment via axioms.  
      \item Even if we have complete knowledge, it often turns out that describing every possible case via logic
            formulae is just unwieldy.  Consider the following formula:
            \\[0.2cm]
            \hspace*{1.3cm}
            $\forall x:\bigl( \mathtt{bird}(x) \rightarrow \mathtt{flies}(x) \bigr)$
            \\[0.2cm]
            The problem with this formula is that although it appears to be common sense, there are a number of counter 
            examples:
            \begin{enumerate}
            \item Penguins, Emus, and ostriches don't fly.

                  However, if we put a penguin into a plane, it turns out the penguin will fly.
            \item Birds that are too young do not fly.
            \item Birds with clipped wings do not fly.
            \item $\cdots$
            \end{enumerate}
            Trying to model all eventualities with logic has turned out to be too unwieldy to be practical.
      \item In real life situations we often deal with \emph{\color{blue}uncertainty}.  Classical logic does not
            perform well when it has to deal with  uncertainties.
      \end{enumerate}
\item \emph{Artificial Intelligence} is the science of creating machines that \underline{act rationall}y.

      All we really want is to build machines that, given the knowledge we have, try to 
      \emph{\color{blue}optimize the expected results}:  In our world, there is lots of uncertainty.  We cannot
      hope to create 
      machines that always make the decisions that turn out to be optimal.  What we can
      hope is to create machines that will make decisions that turn out to be good on average.  For
      example, suppose we try to create a program for asset management:  We cannot hope to build a
      machine that always buys the best company share in the stock market.  Rather, our goal should
      be to build a program that maximizes our expected profits in the long term.

      It has turned out that the main tool needed for this approach is not mathematical logic but
      rather \href{https://en.wikipedia.org/wiki/Numerical_analysis}{numerical analysis} and
      \href{https://en.wikipedia.org/wiki/Mathematical_statistics}{mathematical statistics}.  The shift 
      from logic to numerical analysis and statistics has been the most important reason for the success of
      Artificial Intelligence in the recent years.  Another important factor is the 
      \emph{\color{blue}enhanced performance of modern hardware}. 
\end{enumerate}
Now that we have clarified the notion of artificial intelligence, we should set its goals.  As we
can never achieve more than what we aim for, we have every reason to be ambitious here.  For example, my
personal vision of Artificial Intelligence goes like this: 
Imagine 70 years from now you (not feeling too well) have a conversation with
\href{https://en.wikipedia.org/wiki/Siri}{Siri}.  Instead of asking Siri for the
best graveyard in the vicinity, you think about all the sins you have committed.  As Siri has
accompanied you for your whole life, she knows about these sins better than you.  Hence,  
the conversation with Siri works out as follows: 
\\[0.2cm]
\begin{tabular}[t]{ll}
\textbf{You (with trembling voice):}           & Hey Siri, does God exist?                   \\[0.2cm]
\textbf{Siri (with the voice of Darth Vader):} & Your voice seems troubled, let me think $\cdots$ \\
After a small pause which almost drains \\
the battery of your phone completely, \\
Siri gets back with a soothing announcement: \\           
                                     & You don't have to worry any more, I have fixed the problem.  \\
                                     & He is dead now.  
\end{tabular}
\\[0.2cm]
May \href{https://en.wikipedia.org/wiki/The_Force_(Star_Wars)}{The Force} be with us on achieving our goals!
\pagebreak

\section{Literature}
The main sources of these lecture notes are the following:
\begin{enumerate}
\item A course on artificial intelligence that was offered on the \textsc{edX} platform.  The course
      materials are available at  
      \\[0.2cm]
      \hspace*{1.3cm}
      \href{http://ai.berkeley.edu/home.html}{http://ai.berkeley.edu/home.html}.
\item The book
      \\[0.2cm]
      \hspace*{1.3cm}
      \emph{Introduction to Artificial Intelligence}
      \\[0.2cm]
      written by Stuart Russel and Peter Norvig \cite{russell:2009}.
\item A course on artificial that is offered on \href{https://www.udacity.com}{Udacity}.  The title of the
      course is
      \\[0.2cm]
      \hspace*{1.3cm}
      \href{https://www.udacity.com/course/intro-to-artificial-intelligence--cs271}{Intro to Artificial Intelligence}
      \\[0.2cm]
      and the course is given by \href{https://en.wikipedia.org/wiki/Peter_Norvig}{Peter Norvig}, who is
      director of research at Google and \href{https://en.wikipedia.org/wiki/Sebastian_Thrun}{Sebastian Thrun},
      who is the chairman of \href{https://www.udacity.com}{Udacity}.
\end{enumerate}
The programs presented in these lecture notes have been tested with version 2.6 of 
\href{https://randoom.org/Software/SetlX}{\textsc{SetlX}}.


%%% Local Variables:
%%% mode: latex
%%% TeX-master: "artificial-intelligence"
%%% End:
